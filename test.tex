\documentclass[a5paper,14pt]{memoir}
\usepackage[margin=0.5cm]{geometry}
\usepackage{enumitem}
\setlist[enumerate]{left=0em}
\setlist[itemize]{left=0em}

\usepackage{fontspec}
\usepackage{unicode-math}
\setmainfont{Fira Sans}
\setmathfont{Latin Modern Math}

\usepackage{amsmath}

\title{1.4 Predicates and Quantifiers}
\author{}
\date{}

\begin{document}

\maketitle

\[
  ∀X [ ∅ ∉ X ⇒ ∃f:X ⟶  ⋃ X\ ∀A ∈ X (f(A) ∈ A ) ]\]


\begin{enumerate}
  \item (*) Show that $¬ and ∧$ form a functionally complete collection of logical operators. [Hint: Firt use a De Morgan law to show that $p ∨ q$ is logically equivalent to $¬(¬p ∧ ¬q)$.]
        \begin{itemize}
          \item De Morgans law shows that the negation of $¬(¬p ∧ ¬q)$ is $p ∨ q$ and vice versa. Each occurrence of $∨$ can therefore be expressed with $¬$ and $∧$. Applying this to the disjunctive normal form which can be used to express any compund proposition gives us
          \item $p1 ∨ p2 ∨ ··· ∨ pn$
          \item $\bigwedge_{r=0}^{1} \bigwedge_{s=0}^{1}\bigwedge_{n=1}^{4}\bigvee_{i=1}^{2}\bigvee_{j=1}^{2} p(2r + i, 2s + j, n)$
          \item $¬(¬p1 ∧ ¬p2 ∧ ··· ∧ ¬pn)$
          \item therefore $¬$ and $∧$ are sufficient to express any compound proposition.
        \end{itemize}
  \item First item
        \begin{itemize}
          \item First item
                $$
                  \begin{array}{|c c|c|c|}
                    p & q & p ∨ q & p ∧ (p ∨ q) \\
                    \hline
                    T & T & T     & T           \\
                    T & F & T     & T           \\
                    F & T & T     & F           \\
                    F & F & F     & F           \\
                  \end{array}
                $$
          \item Second item
                \begin{equation}
                  y = \frac{-b \pm \sqrt{b^2 - 4ac}}{2a}
                \end{equation}
          \item Third item
        \end{itemize}
  \item Second item
        \begin{align*}
          p ∧ (p → q) → q & ≡ ¬(p ∧ (¬p ∨ q)) ∨ q                    \\
                          & ≡ (\neg p \lor \neg (\neg p \lor q)) ∨ q \\
                          & ≡ ¬p ∨ ((p ∧ ¬q) ∨ q)                    \\
                          & ≡ ¬p ∨ ((q ∨ p) ∧ (q ∨ ¬q))              \\
                          & ≡ ¬p ∨ ((q ∨ p) ∧ T)                     \\
                          & ≡ ¬p ∨ q ∨ p                             \\
                          & ≡ T ∨ q                                  \\
                          & ≡ T
        \end{align*}
  \item Third item
\end{enumerate}

\end{document}
